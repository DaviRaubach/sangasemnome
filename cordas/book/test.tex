% Created 2024-01-17 Wed 11:03
% Intended LaTeX compiler: pdflatex
\documentclass[11pt]{article}
\usepackage[utf8]{inputenc}
\usepackage[T1]{fontenc}
\usepackage{graphicx}
\usepackage{longtable}
\usepackage{wrapfig}
\usepackage{rotating}
\usepackage[normalem]{ulem}
\usepackage{amsmath}
\usepackage{amssymb}
\usepackage{capt-of}
\usepackage{hyperref}
\usepackage{nopageno}
\usepackage{bicaption}
\setcounter{secnumdepth}{0}
\author{Davi Raubach Tuchtenhagen}
\date{\today}
\title{}
\hypersetup{
 pdfauthor={Davi Raubach Tuchtenhagen},
 pdftitle={},
 pdfkeywords={},
 pdfsubject={},
 pdfcreator={Emacs 29.1 (Org mode 9.6.6)}, 
 pdflang={Portuges}}
\begin{document}



\section{Apresentação}
\label{sec:orgdab6f90}

Substâncias sem nome é um dos resultados de uma pesquisa artística que investiga a imaginação criadora em música inspirada em gaston bachelard. a imaginação da liquidez do material musical motiva a composição e pode motivar a performance e escuta da peça. os complexos espectrais utilizados estão sempre a se formar e se deformar, sempre em devir. assim, a peça convida orquestra e ouvintes a imergirem nesta ambiência sonora de temperamento e andamento fluidos.

\begin{verbatim}




  \version "2.25.11"
  \include "org_stylesheet.ily"



  \paper {
  % page-breaking = #ly:one-line-breaking
    page-breaking = #ly:minimal-breaking
}


paragraph = "Substâncias sem nome é um dos resultados de uma pesquisa artística que investiga a imaginação criadora em música inspirada em gaston bachelard. a imaginação da liquidez do material musical motiva a composição e pode motivar a performance e escuta da peça. os complexos espectrais utilizados estão sempre a se formar e se deformar, sempre em devir. assim, a peça convida orquestra e ouvintes a imergirem nesta ambiência sonora de temperamento e andamento fluidos."

  \markup {
    \fill-line {
      \override #'(line-width . 45)
      \dir-column {\justify-string {\paragraph}}

      \override #'(line-width . 45)
      \dir-column {\justify-string{\paragraph}}
      }
    }


\end{verbatim}

\begin{center}
\includegraphics[width=.9\linewidth]{columns.pdf}
\end{center}


\section{Notas de performance}
\label{sec:org650add5}

\begin{itemize}
\item Todos os naipes apresentam divisi a dois

\item Não há necessidade de afinar os quartos de tom precisamente.

\item Trêmolos: sempre o mais rápido possível

\item Indicações de posição de arco

st.: sul tasto

sp.: sul ponticello

esp.: extremamente sul ponticello

ord.: posição ordinária
\end{itemize}
\end{document}